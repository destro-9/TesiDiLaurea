% Load variables
\newcommand{\myUni}{Università degli Studi di Padova}
\newcommand{\myDepartment}{Dipartimento di Matematica ``Tullio Levi-Civita''}
\newcommand{\myFaculty}{Corso di Laurea in Informatica}
\newcommand{\myTitle}{Excelgen: generazione automatica della reportistica nei più comuni formati documentali}
\newcommand{\myDegree}{Tesi di Laurea Triennale}
\newcommand{\profTitle}{Prof.}
\newcommand{\myProf}{Ranzato Francesco}
\newcommand{\graduateTitle}{Laureando}
\newcommand{\myName}{Destro Stefano}
\newcommand{\myStudentID}{1229139}
\newcommand{\myAA}{2023-2024}
\newcommand{\myLocation}{Padova}
\newcommand{\myTime}{Settembre 2024}
% Acronyms
\newacronym{api}{API}{Application Program Interface}
\newacronym{sdk}{SDK}{Software Development Kit}
\newacronym{uml}{UML}{Unified Modeling Language}
\newacronym{tsa}{TSA}{Termine solo acronimo}
\newacronym{mvc}{MVC}{Model View Controller}
\newacronym{drf}{DRF}{Django REST Framework}
\newacronym{orm}{ORM}{Object-Relational Mapping}
\newacronym{dom}{DOM}{Document Object Model}
\newacronym{pmi}{PMI}{piccole e medie imprese}

% Glossary
\newglossaryentry{apig}{
    name={API},
    text={Application Program Interface},
    sort=api,
    description={In informatics, an API is a set of procedures available to programmers, typically grouped to form a toolkit for a specific task within a program. Its purpose is to provide an abstraction, usually between hardware and the programmer or between low-level and high-level software, simplifying the programming process}
}

\newglossaryentry{sdkg}{
    name={SDK},
    text={Software Development Kit},
    sort=sdk,
    description={A Software Development Kit (SDK) is a collection of development tools in one installable package, facilitating application creation by providing a compiler, debugger, and sometimes a software framework. SDKs are typically specific to a hardware platform and operating system combination. Many application developers use specific SDKs to enable advanced functionalities such as advertisements, push notifications, etc}
}

\newglossaryentry{umlg}{
    name={UML},
    text={Unified Modeling Language},
    sort=uml,
    description={In software engineering, Unified Modeling Language (UML) is a modeling and specification language based on the object-oriented paradigm. UML serves as a "lingua franca" in the object-oriented design and programming community. Much of the industry literature uses UML to describe analytical and design solutions in a concise and understandable way for a broad audience}
}

\newglossaryentry{TermineSenzaAcronimo}{
    name={Nome del termine},
    sort=termine senza acronimo,
    description={Descrizione}
}

\newglossaryentry{ormg}{
	name={ORM},
	sort=orm,
	text={Object-Relational Mapping},
	description={Tecnica di programmazione che permette di interagire con un database relazionale utilizzando oggetti di programmazione anziché scrivere direttamente query SQL. L'ORM mappa le classi del linguaggio di programmazione ai modelli del database, consentendo di manipolare i dati del database attraverso metodi e proprietà degli oggetti, semplificando la gestione delle operazioni di Create, Read, Update, Delete}
}

\newglossaryentry{domg}{
	name={DOM},
	sort=dom,
	text={Document Object Model},
	description={Rappresentazione ad albero di un documento HTML o XML, permette ai linguaggi di programmazione di manipolare la struttura, il contenuto e lo stile di una pagina web in modo dinamico}
}

\newglossaryentry{virtdom}{
	name={Virtual DOM},
	sort=virtdom,
	text={Virtual Document Object Model},
	description={copia leggera e in-memory del DOM reale, utilizzata principalmente da librerie come React per ottimizzare le operazioni di aggiornamento dell'interfaccia utente. Le modifiche vengono fatte sul Virtual DOM, e solo le differenze rispetto al DOM reale vengono applicate, migliorando così l'efficienza del rendering}
}

\newglossaryentry{pip}{
	name={pip},
	sort=pip,
	text={pip},
	description={Gestore di pacchetti ufficiale di Python, usato per installare, aggiornare e rimuovere pacchetti e librerie. Gestisce le dipendenze e facilita l'integrazione di nuovi pacchetti nei progetti Python}
}

\newglossaryentry{queryset}{
	name={QuerySet},
	sort=queryset,
	text={QuerySet},
	description={rappresentazione astratta di una query al database che restituisce una collezione di oggetti del modello su cui è stato invocato. È valutato in modo \textit{lazy}, consentendo di costruire e filtrare dinamicamente le query senza eseguire nell'immediato le operazioni sul database. I QuerySet supportano operazioni come filtraggio, ordinamento e aggregazione.}
}

\newglossaryentry{playground}{
	name={playground},
	sort=playground,
	text={playground},
	description={ambiente semplificato e isolato che emula le funzionalità di un sistema reale per consentire lo sviluppo, la sperimentazione di codice senza effetti collaterali su sistemi di produzione. Questi ambienti sono utilizzati per simulare scenari, verificare il comportamento di software e provare nuove idee in un contesto controllato e sicuro.}
}

\newglossaryentry{pandas}{
	name={Pandas},
	sort=Pandas,
	text={Pandas},
	description={libreria Python per l'analisi e la manipolazione di dati, offre strutture dati come DataFrame e Series. Fornisce funzionalità avanzate per l'elaborazione e l'analisi di dataset strutturati. Nell'ecosistema Django, Pandas può essere utilizzata efficacemente per elaborare i dati estratti dai modelli Django, trasformando queryset in DataFrame per analisi avanzate o per preparare i dati prima di renderizzarli o esportarli.}
}

% Define custom colors
\definecolor{hyperColor}{HTML}{0B3EE3}
\definecolor{tableGray}{HTML}{F5F5F7}
\definecolor{veryPeri}{HTML}{6667ab}

% Set line height
\linespread{1.5}

% Custom hyphenation rules
\hyphenation {
    data-base
    al-go-rithms
    soft-ware
}

% Images path
\graphicspath{{img/}}

% Force page color, as some editors set a grayish color as default
\pagecolor{white}

% Better spacing for footnotes
\setlength{\skip\footins}{5mm}
\setlength{\footnotesep}{5mm}

\setlength{\headheight}{14.5pt}
\addtolength{\topmargin}{-2.45pt}

% Add a subscript G to a glossary entry
\newcommand{\glox}{\textsubscript{\textbf{\textit{G}}}}

% Improvements the paragraph command
\titleformat{\paragraph}
{\normalfont\normalsize\bfseries}{\theparagraph}{1em}{}
\titlespacing*{\paragraph}
{0pt}{3.25ex plus 1ex minus .2ex}{1.5ex plus .2ex}

% Define use case environment
\newcounter{usecasecounter} % define a counter
\setcounter{usecasecounter}{0} % set the counter to some initial value
% Parameters
% #1: ID
% #2: Nome
\newenvironment{usecase}[2]{
    \renewcommand{\theusecasecounter}{\usecasename #1}  % this is where the display of the counter is overwritten/modified
    \refstepcounter{usecasecounter} % increment counter
    \vspace{2em}
    \par \noindent % start new paragraph
    {\normalsize \textbf{\usecasename #1: #2}} % display the title before the content of the environment is displayed
    \vspace{.5em}
}{
    \medskip
}
\newcommand{\usecasename}{UC}
\newcommand{\usecaseactors}[1]{\textbf{\\Attori Principali:} #1}
\newcommand{\usecasepre}[1]{\textbf{\\Precondizioni:} #1}
\newcommand{\usecasedesc}[1]{\textbf{\\Descrizione:} #1}
\newcommand{\usecasepost}[1]{\textbf{\\Postcondizioni:} #1}
\newcommand{\usecasealt}[1]{\textbf{\\Scenario Alternativo:} #1}

% Define risks environment
\newcounter{riskcounter} % define a counter
\setcounter{riskcounter}{0} % set the counter to some initial value
% Parameters
% #1: Title
\newenvironment{risk}[1]{
    \refstepcounter{riskcounter} % increment counter
    \par \noindent % start new paragraph
    \textbf{\arabic{riskcounter}. #1} % display the title before the content of the environment is displayed
}{
    \par\medskip
}
\newcommand{\riskname}{Rischio}
\newcommand{\riskdescription}[1]{\textbf{\\Descrizione:} #1.}
\newcommand{\risksolution}[1]{\textbf{\\Soluzione:} #1.}

% Apply fancy styling to pages
\pagestyle{fancy}
\fancyhf{}
\fancyhead[L]{\leftmark} % Places Chapter N. Chatper title on the top left
\fancyfoot[C]{\thepage} % Page number in the center of the footer

% Adds a blank page while increasing the page number
\newcommand\blankpage{ 
\clearpage
    \begingroup
    \null
    \thispagestyle{empty}
    \hypersetup{pageanchor=false}
    \clearpage
\endgroup
}

% Adds a blank page while increasing the page number
\newcommand\blankpagewithnumber{ 
  \clearpage
  \thispagestyle{plain} % Use plain page style to keep the page number
  \null
  \clearpage
}

% Increase page numbering
\newcommand\increasepagenumbering{
    \addtocounter{page}{+1}
}

% Make glossaries and bibliography
\makeglossaries
% Redefine the format for the glossary entries to be italic
\renewcommand*{\glstextformat}[1]{\textit{#1}\glox}
%\glsaddall

\bibliography{references/bibliography}
\defbibheading{bibliography} {
    \cleardoublepage
    \phantomsection
    \addcontentsline{toc}{chapter}{\bibname}
    \chapter*{\bibname\markboth{\bibname}{\bibname}}
}

% Code blocks handling w/ table of codes
\makeatletter
\ifdefined\NR@chapter
  \expandafter\@firstoftwo
\else
  \expandafter\@secondoftwo
\fi{\patchcmd\NR@chapter}{\patchcmd\@chapter}
  {\addtocontents{lot}{\protect\addvspace{10\p@}}}
  {\addtocontents{lot}{\protect\addvspace{10\p@}}%
   \addtocontents{lol}{\protect\addvspace{10\p@}}}
  {}{}
\makeatother

\renewcommand\listingscaption{Codice}
\renewcommand\listoflistingscaption{Elenco dei codici sorgenti}
\counterwithin*{listing}{chapter}
\renewcommand\thelisting{\thechapter.\arabic{listing}}

% Set up hyperlinks
\hypersetup{
    colorlinks=true,
    linktocpage=true,
    pdfstartpage=1,
    pdfstartview=,
    breaklinks=true,
    pdfpagemode=UseNone,
    pageanchor=true,
    pdfpagemode=UseOutlines,
    plainpages=false,
    bookmarksnumbered,
    bookmarksopen=true,
    bookmarksopenlevel=1,
    hypertexnames=true,
    pdfhighlight=/O,
    allcolors = hyperColor
}

% Set up captions
\captionsetup{
    tableposition=top,
    figureposition=bottom,
    font=small,
    format=hang,
    labelfont=bf
}

% When draft mode is on, the hyperlinks are removed. Useful when printing the document. To enable/disable, uncomment/comment the command
% \hypersetup{draft}

% prevents cleaning up the cache at the end of the run (needed to keep the unused caches, generated by other editions)
\makeatletter
\renewcommand*{\minted@cleancache}{}
\makeatother

% Break lines in code blocks whe using inputminted
\setminted{breaklines}