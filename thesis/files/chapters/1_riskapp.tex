\chapter{RiskApp}
\label{chap:introduzione}



\section{Storia}
Fondata nel 2015 da un gruppo di studenti dell'Università Ca' Foscari di Venezia, RiskApp (logo in figura \ref{fig:riskapp}) è una \textit{startup insurtech} con sede a Conselve, in provincia di Padova. La società si è rapidamente affermata nel settore della trasformazione digitale e dell'analisi avanzata del rischio per il comparto assicurativo.

\begin{figure}[H]
	\centering
	\includegraphics[alt={Logo docx, xlsx e pdf}, width=0.5\columnwidth]{img/riskapp_logo.png}
	\caption{Logo RiskApp}
	\label{fig:riskapp}
\end{figure}

\subsection{Insurance Advisor}
In parallelo a RiskApp opera Insurance Advisor, una società di consulenza assicurativa. Sebbene le due aziende mantengano una propria identità e un proprio campo d'azione, lavorano in sinergia per offrire un servizio completo ai propri clienti.
Insurance Advisor è una società di consulenza assicurativa che supporta privati e aziende nella scelta delle coperture assicurative più adatte alle loro specifiche necessità. L'approccio verso le aziende è altamente personalizzato e basato su un'analisi approfondita delle esigenze del cliente e su una consulenza indipendente.
Grazie alle soluzioni digitali di RiskApp, Insurance Advisor può offrire ai propri clienti un’analisi del rischio più precisa e basata su dati reali, migliorando così la qualità delle loro consulenze.

\section{Obiettivi}
L'azienda integra tecnologie avanzate come \textit{machine learning}, i \textit{big data} e il \textit{cloud computing} nelle sue soluzioni. Queste tecnologie permettono a RiskApp di offrire una gamma di servizi che supportano le compagnie assicurative e gli intermediari nel loro percorso di digitalizzazione e nella gestione delle esigenze assicurative delle \gls{pmi}. L'obiettivo di RiskApp è fornire strumenti che non solo migliorano l'efficienza operativa, ma che consentono anche una gestione proattiva e a 360 gradi del rischio, rendendo le aziende più resilienti di fronte agli imprevisti.

L'azienda ha sviluppato una piattaforma digitale avanzata che consente alle compagnie assicurative di migliorare i propri processi e agli intermediari di gestire con maggiore precisione e velocità le esigenze dei clienti.

\section{Clienti}
RiskApp offre una gamma di soluzioni che si rivolgono principalmente a due categorie di clienti:
\begin{itemize}
	\item \textbf{Compagnie assicurative:} RiskApp aiuta le compagnie assicurative nel processo di digitalizzazione anticipando per quanto possibile le richieste dei clienti;
	\item \textbf{Intermediari assicurativi:} la piattaforma fornisce strumenti di analisi utili per offrire consulenze più mirate e di individuare le coperture assicurative più adatte alle esigenze assicurative dei clienti.
\end{itemize}

\section{Piattaforme e tecnologie}
Al cuore delle soluzioni di RiskApp vi è una piattaforma tecnologica che sfrutta l'intelligenza artificiale per analizzare grandi quantità di dati e generare previsioni accurate sui rischi. Questa piattaforma è progettata per essere altamente scalabile e flessibile, in modo da poter essere adattata alle diverse esigenze delle compagnie assicurative e dei loro intermediari.

\begin{itemize}
	\item \textbf{Intelligenza artificiale e \textit{Big Data}:} RiskApp utilizza algoritmi avanzati di \textit{machine learning} per analizzare dati storici e attuali, identificando pattern che possono indicare potenziali rischi. Questo approccio permette di prevedere eventi come catastrofi naturali o variazioni significative nel mercato, consentendo ai clienti di adottare misure preventive efficaci;
	\item \textbf{\textit{Cloud Computing}:} la piattaforma di RiskApp è basata su un'infrastruttura \textit{cloud}, che garantisce un accesso sicuro e continuo ai dati e agli strumenti di analisi. Questo consente una gestione centralizzata e in tempo reale delle informazioni, facilitando la collaborazione tra le diverse parti coinvolte nel processo assicurativo;
	\item \textbf{Analisi in tempo reale}: la piattaforma consente il monitoraggio continuo delle variabili di rischio, offrendo aggiornamenti in tempo reale. Questa funzionalità è particolarmente utile per le \gls{pmi} che devono gestire rischi non statici e in continua evoluzione.
\end{itemize}

\subsection{Esperienza personale}
Data la mia esperienza in azienda, queste sono le tecnologie adottate che ho avuto modo di osservare:
\begin{itemize}
	\item \textbf{Computer:} nello specifico portatili Apple con il loro sistema operativo proprietario;
	\item \textbf{Microsoft Office 365:} servizio Microsoft in abbonamento, gli strumenti prevalentemente usati sono Word, Excel e Outlook;
	\item \textbf{Slack:} piattaforma analoga a Microsoft Teams, utile per le conversazioni interne all'azienda e per lo scambio veloce di allegati;
	\item \textbf{GitHub:} piattaforma di \textit{hosting} per progetti software basata su Git, che offre versionamento del codice, gestione collaborativa tramite \textit{pull request}, integrazione continua e un \textit{issue tracking system} per il monitoraggio di \textit{bug} e richieste;
	\item \textbf{PyCharm Professional:} IDE avanzato per lo sviluppo Python, con funzionalità di \textit{debug}, supporto per \textit{framework web} e strumenti di \textit{refactoring} e \textit{testing};
	\item \textbf{Jira:} una piattaforma di gestione dei progetti e \textit{issue tracking}, ampiamente utilizzata per il monitoraggio delle attività, la gestione dei \textit{bug} e la pianificazione agile, con strumenti di \textit{reporting} e integrazione continua;
	\item \textbf{Visual Studio Code:} editor di codice open-source estensibile, che supporta numerosi linguaggi di programmazione, offre debugging integrato, terminale, e un ricco ecosistema di estensioni;
	\item \textbf{Docker:} una piattaforma per la creazione e gestione di container, che consente agli sviluppatori di automatizzare la distribuzione delle applicazioni e garantire consistenza tra ambienti di sviluppo, \textit{testing} e produzione;
	\item \textbf{WebStorm Professional:} IDE per lo sviluppo JavaScript e TypeScript, con strumenti integrati per il debugging, il testing, il refactoring e il supporto avanzato per framework front-end come React.
\end{itemize}

RiskApp per sviluppare i suoi applicativi usa più linguaggi di programmazione, \textit{framework} e librerie in base alle proprie necessità.

Le principali tecnologie per \textit{frontend} e \textit{backend} che ho avuto modo di vedere e/o discutere sono:
\begin{itemize}
	\item \textbf{Django:} framework web in Python che facilita lo sviluppo rapido di applicazioni, fornendo un'architettura \gls{mvc}, gestione dei database tramite \gls{ormg}, autenticazione e pannello di amministrazione integrati;
	\item \textbf{\gls{drf}:} toolkit per la creazione di API RESTful in Django, che include gestione delle autenticazioni, serializzazione dei dati e strumenti per il versionamento;
	\item \textbf{React:} libreria JavaScript per la costruzione di interfacce utente interattive e componenti riutilizzabili, basata su un modello a componenti e il concetto di \gls{virtdom} che gestisce le operazioni sul \gls{domg} per ottimizzare le prestazioni di rendering;
	\item  \textbf{Celery}: libreria Python per l'esecuzione asincrona di task distribuiti, spesso utilizzata con Django per gestire operazioni a lungo termine come invio di email, elaborazione di dati o operazioni di background.
\end{itemize} 

\subsection{Servizi offerti}
Insieme, RiskApp e Insurance Advisor offrono una gamma di servizi che copre tutte le fasi della gestione del rischio e della consulenza assicurativa:
\begin{itemize}
	\item \textbf{Analisi del rischio su misura:} RiskApp fornisce gli strumenti tecnologici per l'analisi dei dati e la previsione dei rischi, mentre Insurance Advisor si occupa di interpretare i dati e quindi consigliare le coperture assicurative più adeguate;
	\item \textbf{Gestione del rischio continuativa:} Questo servizio è particolarmente utile per le aziende che operano in settori ad alta volatilità o che sono esposte a rischi significativi, come quelli legati alle catastrofi naturali;
	\item \textbf{Consulenza assicurativa:} Insurance Advisor, offre una consulenza mirata supportando i clienti nella fase di scelta delle polizze. Questo approccio consente di ridurre i costi e di evitare coperture ridondanti o non necessarie;
	\item \textbf{Supporto post-vendita:} dopo la sottoscrizione delle polizze, viene offerto supporto nella gestione dei sinistri e nella revisione periodica delle coperture. La piattaforma di RiskApp fornisce dati aggiornati che possono essere utilizzati per rinegoziare i termini delle polizze o per adattare le coperture alle nuove condizioni di rischio.
\end{itemize}

\section{Organizzazione del testo}
\begin{description}
    \item[{\hyperref[chap:stage_desc]{Il secondo capitolo}}] descrive ...
    
    \item[{\hyperref[chap:descrizione-stage]{Il terzo capitolo}}] approfondisce ...
    
    \item[{\hyperref[chap:analisi-requisiti]{Il quarto capitolo}}] approfondisce ...
    
    \item[{\hyperref[chap:progettazione-codifica]{Il quinto capitolo}}] approfondisce ...
    
    \item[{\hyperref[chap:verifica-validazione]{Il sesto capitolo}}] approfondisce ...
    
    \item[{\hyperref[chap:conclusioni]{Nel settimo capitolo}}] descrive ...
\end{description}

Riguardo la stesura del testo, relativamente al documento sono state adottate le seguenti convenzioni tipografiche:
\begin{itemize}
	\item gli acronimi, le abbreviazioni e i termini ambigui o di uso non comune menzionati vengono definiti nel glossario, situato alla fine del presente documento;
	\item per la prima occorrenza dei termini riportati nel glossario viene utilizzata la seguente nomenclatura: \gls{domg};
	\item i termini in lingua straniera o facenti parti del gergo tecnico sono evidenziati con il carattere \textit{corsivo}.
\end{itemize}

\begin{listing}[H]
\begin{minted}{c}
#include <stdio.h>
int main() {
    print("Hello, world!");
    return 0;
}
\end{minted}
\caption{Example of code}
\label{listing:a}
\end{listing}

\newpage