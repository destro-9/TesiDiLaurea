\chapter{Introduzione}
\label{chap:introduzione}

\begin{figure}[H]
    \centering
    \includegraphics[alt={Logo docx, xlsx e pdf}, width=1\columnwidth]{img/format_logos.jpg}
    \caption{Loghi principali formati documentali}
    \label{fig:logos}
\end{figure}

Introduzione al contesto applicativo.

Lorem Figure \ref{fig:entanglement}

Esempio di utilizzo di un termine nel glossario \gls{api}.

Esempio di citazione in linea \cite{site:agile-manifesto}.

Esempio di citazione nel piè di pagina citazione\footcite{womak:lean-thinking}
Introduzione all'idea dello stage\footcite{article:spooky}.

\lipsum[1-2]

\section{L'azienda}

\begin{figure}[H]
    \centering
    \includegraphics[alt={Logo docx, xlsx e pdf}, width=0.5\columnwidth]{img/riskapp_logo.png}
    \caption{Loghi principali formati documentali}
    \label{fig:riskapp}
\end{figure}

RiskApp (logo in figura \ref{fig:riskapp}) è una startup insurtech innovativa fondata nel 2015 da un gruppo di studenti dell'Università Ca' Foscari di Venezia. Con sede a Conselve (PD), l'azienda si è affermata come una delle realtà più all'avanguardia in Italia nel campo della trasformazione digitale e dell'analisi avanzata del rischio per il settore assicurativo. RiskApp si distingue per lo sviluppo di piattaforme tecnologicamente avanzate che supportano le compagnie assicurative nel loro percorso di digitalizzazione e forniscono agli intermediari strumenti sofisticati per gestire le esigenze assicurative delle piccole e medie imprese. Sfruttando tecnologie come l'intelligenza artificiale, i big data e il cloud computing. RiskApp offre soluzioni che spaziano dalla previsione di catastrofi naturali al monitoraggio online, consentendo una gestione del rischio a 360 gradi.





\section{L'idea}
L'idea alla base dello stage nasce da un'esigenza concreta di RiskApp di ottimizzare e standardizzare il proprio processo di generazione di report. L'azienda, nel suo ruolo di fornitore di soluzioni insurtech avanzate, offre ai propri clienti la possibilità di scaricare report dettagliati in vari formati, tra cui fogli di calcolo, documenti di testo e PDF. Questi report variano significativamente in contenuto e struttura a seconda dei dati specifici e del servizio utilizzato. Fino ad ora, RiskApp ha gestito questa complessità attraverso l'uso di script per casi specifici anche isolati, un approccio che, seppur funzionale, presentava limitazioni in termini di scalabilità ed efficienza. Il bisogno di effettuare modifiche a cascata fra i vari script, ha evidenziato la necessità di sviluppare una soluzione più flessibile e dinamica. Da qui è nata l'idea di creare una libreria Python universale, capace di generare report in vari formati a partire da dati e configurazioni di input standardizzate. Questo progetto mira non solo a semplificare e velocizzare il processo di creazione dei report, ma anche a fornire una base solida per future espansioni e personalizzazioni, allineandosi con la filosofia di innovazione continua di RiskApp.

\section{Organizzazione del testo}
\begin{description}
    \item[{\hyperref[chap:processi-metodologie]{Il secondo capitolo}}] descrive ...
    
    \item[{\hyperref[chap:descrizione-stage]{Il terzo capitolo}}] approfondisce ...
    
    \item[{\hyperref[chap:analisi-requisiti]{Il quarto capitolo}}] approfondisce ...
    
    \item[{\hyperref[chap:progettazione-codifica]{Il quinto capitolo}}] approfondisce ...
    
    \item[{\hyperref[chap:verifica-validazione]{Il sesto capitolo}}] approfondisce ...
    
    \item[{\hyperref[chap:conclusioni]{Nel settimo capitolo}}] descrive ...
\end{description}

Riguardo la stesura del testo, relativamente al documento sono state adottate le seguenti convenzioni tipografiche:
\begin{itemize}
	\item gli acronimi, le abbreviazioni e i termini ambigui o di uso non comune menzionati vengono definiti nel glossario, situato alla fine del presente documento;
	\item per la prima occorrenza dei termini riportati nel glossario viene utilizzata la seguente nomenclatura: \gls{apig};
	\item i termini in lingua straniera o facenti parti del gergo tecnico sono evidenziati con il carattere \textit{corsivo}.
\end{itemize}

\begin{listing}[H]
\begin{minted}{c}
#include <stdio.h>
int main() {
    print("Hello, world!");
    return 0;
}
\end{minted}
\caption{Example of code}
\label{listing:a}
\end{listing}

\newpage